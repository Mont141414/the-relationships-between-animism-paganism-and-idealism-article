This article presents a compelling intersection of philosophy, spirituality, and cultural tradition, seeking to explore these concepts
through an ontological lens, arguing for idealism as a framework through which animistic and pagan worldviews gain coherence.
By synthesizing definitions from philosophical discourse, historical analysis, and contemporary practices, the discussion aims to
illuminate how these systems collectively challenge materialist paradigms and reframe humanity’s relationship with existence.

While the author's background in biology diverges from formal training in philosophy or theology, this interdisciplinary approach
underscores the universality of qualitative experience and invites the reader to reconsider the limitations that are present in popular
thoughts, engaging instead with alternative epistemologies that prioritize relationality and presence.

