\section{What is it like to be a bat?}
"What is it like to be a bat?" is the name of an article written by the American philosopher
Thomas Nagel and published in 1974. The core argument of the book is that we cannot perceive
how it's like to be a bat, as in, we cannot feel like a bat, nor any other perspective that
is not our own. As Nagel explains:

\begin{center}
    \itshape
    \parbox{0.7\textwidth}{
    "In so far as I can imagine this (which is not very far), it tells me only what it would
    be like for me to behave as a bat behaves. But that is not the question. I want to know
    what it is like for a bat to be a bat. Yet if I try to imagine this, I am restricted to
    the resources of my own mind, and those resources are inadequate to the task."\\
    \normalfont - Thomas Nagel, "What is it like to be a bat?" \cite{Nagel2024}
    }
\end{center}

Nagel's argument centres on the irreducible nature of subjectivity as the notion that
consciousness is not merely a sequence of physical processes, but a first-person and lived
experience unique to each being. By emphasizing the impossibility of "feeling" another
entity's subjective reality, he highlights a fundamental epistemic boundary: we cannot directly
inhabit the experiential world of a bat (or any other being) because their consciousness is shaped
by perceptual, cognitive, and sensory frameworks alien to our own. While Nagel does not
explicitly address animism, his framework nonetheless resonates with animistic perspectives, as
both grapple with the inaccessibility of "other" subjectivenesses. This tension sets the stage
for exploring how both compare when given an idealistic framework.

\section{Subjectiveness in an idealistic reality}
Considering the reality as ultimately qualitative, we could stretch Nagel's perspective on
subjectivity. It is argued, like how we previously discussed, that our perception is not the
same one as the bat's perception, nor my perception is the same as your perception, but a question
that could be brought up:

\begin{itemize}
    \item What about the rock?
\end{itemize}

When I say rock, I do not mean Dwayne "The Rock" Johnson, I mean literally a rock, a naturally
occurring aggregate of minerals. According to Nagel, consciousness occurs...

\begin{center}
    \itshape
    \parbox{0.7\textwidth}{
    "[...] at many levels of animal life, though we cannot be sure of its presence in the simpler
    organisms, and it is very difficult to say in general what provides evidence of it. [...] No
    doubt it occurs in countless forms totally unimaginable to us, on other planets in other
    solar systems throughout the universe. But no matter how the form may vary, the fact that an
    organism has conscious experience at all means, basically, that there is something it is like
    to be that organism."\\
    \normalfont - Thomas Nagel, "What is it like to be a bat?" \cite{Nagel2024}
    }
\end{center}

If the experience is something that brings the idea of being like said organism, then in an universe
that has experience and qualities as foundation of reality would in turn also have that characteristic
applied to it as a whole, as well as parts of it. In a nutshell, the rock has the same qualitative
ground that we do, therefore we can ask ourselves a new question:

\begin{itemize}
    \item What is it like to be a rock?
\end{itemize}

\section{Taking care of others}
The rock, in our perspective, is an inanimate object, and we are not wrong in saying that, but we
do assume that because it lacks, for example, pain receptors, that it does not feel pain. The
problem with that assumption is that pain itself, as a quality, is an experience, and the way we
usually assess if an entity is feeling pain is by looking at their movements, sounds, and overall
reactions. The rock, however, does not produce such things, therefore we truly cannot even guess
what it feels. Even though that's the case, it brings an interesting moral dilemma:

\begin{itemize}
    \item Should we even care what inanimate beings feel?
\end{itemize}

In my point of view, we should treat them with respect, for we don't know how it will be like for us
when other animate beings treat us badly after we become inanimate, as in, after we die.