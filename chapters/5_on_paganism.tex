\section{Paganism in the past}
Paganism, referring to any non-Abrahamic religious tradition, as we discussed in the definitions
chapter, is and was very diverse in Europe, covering all parts of it that have been inhabited by
humans, as well as having various different conceptual approaches, such as the dichotomy between
centralized religion (such as the Romans had, with their priesthood and specific rituals) and 
decentralized (such as the case with the Nordics, that relied more on oral tradition and tribal
rites). As it was cited in "The Handbook of Religions in Ancient Europe":

\begin{center}
    \itshape
    \parbox{0.7\textwidth}{
    "The picture of religion in ancient Europe that emerges even from such a thumbnail sketch is
    one of diversity and complexity. The time depth is vast, the evidence – whether archaeological
    or written – is difficult to interpret, and the variety of religions documented by the sources
    is bewildering. For students, scholars and members of the broader public wishing to expand their
    horizons in this field, it is challenging in the extreme to attempt to gain an overall picture of
    religious history before the advent of Christianity."\\
    \normalfont - The Handbook of Religions in Ancient Europe \cite{Christensen2020}
    }
\end{center}

A second example, albeit controversial, is that the diversity in thought also led to the emergence of
monotheistic pagan religions in Europe. Stephen Mitchell, on his book "One God: Pagan Monotheism in the
Roman Empire", discusses some possibilities of monotheistic thoughts in Greece and Rome, Antisthenes
being one of the examples of, according to his opinion, "monotheistic pagans of Europe":

\begin{center}
    \itshape
    \parbox{0.7\textwidth}{
    "Philodemus says that Antisthenes in
    his work Physicus claims that by convention (κατά νόμον) there are many gods, but that
    by nature (κατά φύσιν) there is just one. Cicero presents the Epicurean Velleius as saying
    that according to Antisthenes the popular gods are many, but that by nature, in the nature of things,
    there is just one god. Velleius claims that Antisthenes in taking this position does away with the
    power and the reality of the gods."\\
    \normalfont - Stephen Mitchell \cite{Mitchell2010}
    }
\end{center}

Nonetheless, as time passed and more people converted, willingly or by force, to the christian faith,
more and more of that traditional diversity was lost, until virtually everyone was partially or fully
integrated into said Abrahamic religion.

\section{Modern western paganism}
In the modern era, with the secularization of the world and the disillusionment with the christian
churches and their ideas, there has been a resurgence of non-Abrahamic, pagan traditions, in Europe
and their past colonies. There are three main types of self-declared pagans nowadays in the western
world: reconstructionists, neo-pagans, and atheist pagans.

The reconstructionists, as the name implies, are the ones that try reviving the ancient religious
traditions by reconstructing them with the information available (as it's the case with, for example,
the European Congress of Ethnic Religions (ECER) and the organizations that are part of it).
As it's mentioned by them in their "about" page:

\begin{center}
    \itshape
    \parbox{0.7\textwidth}{
    "The purpose of the ECER is to serve as an international body that will assist Ethnic Religious
    groups in various countries and will oppose discrimination against such groups. By Ethnic Religion,
    we mean religion, spirituality, and cosmology that is firmly grounded in a particular people's
    traditions. In our view, this does not include modern occult or ariosophic theories/ideologies, nor
    syncretic neo-religions."\\
    \normalfont - European Congress of Ethnic Religions \cite{ECER}
    }
\end{center}

Neo-pagans, different from the reconstructionists, do not attempt to follow the ancient ways as how they
were followed, instead they create their own thing by mashing together elements from other religions and
worldviews until it becomes something totally different from the original elements by themselves (as it's
the case with, for example, the Wiccans). According to Ethan Doyle White, in his book "Wicca: History,
Belief and Community in Modern Pagan Witchcraft", he states:

\begin{center}
    \itshape
    \parbox{0.7\textwidth}{
    "Historical research has established that it [Wicca] is a twentieth-century new religious movement that,
    to a significant extent, consists of a patchwork built up from various older sources. [...] This should
    come as no surprise; after all, despite the fact that devout believers often insist against all odds that
    their faith came direct from a divine source free from human influences, in reality all the major
    religious ideologies can be identified as having antecedents. Wicca is no exception, and exists because
    it was able to draw upon a wide array of pre-existing religious, spiritual, and esoteric movements."\\
    \normalfont - Ethan Doyle White \cite{White2015}
    }
\end{center}

As per the definitions used in this article, however, Wicca and the majority of other neo-pagan religions
cannot be considered a pagan religion, since they have direct or indirect influence of Abrahamic religions.
Just as an example stated by White in the case of Wicca:

\begin{center}
    \itshape
    \parbox{0.7\textwidth}{
    "Esoteric ideas could also be found on the fringes of established religion, and it has been suggested that
    Wicca was influenced in some small part by British heterodox Christian groups, in particular forms of
    Anglo-Catholicism, which embraced esoteric ideas, ritualism, and sacral nudity."\\
    \normalfont - Ethan Doyle White \cite{White2015}
    }
\end{center}

Last but not least, "atheist paganism" is an umbrella term for any type of self-declared pagan that does not
believe in the existence of the divine, rather saying that the gods are man-made and represent archetypes of
human psyche. Even though their existence is contradictory, as atheist pagans do not follow the traditions as
religion, some people do identify as "atheist pagans", as it was shown in the book "Godless Paganism: Voices of
Non-Theistic Pagans", in the "Yes, Virginia, I'm a Pagan Atheist" section, by Jeffrey Flagg:

\begin{center}
    \itshape
    \parbox{0.7\textwidth}{
    "I'm an atheist. I'm also Pagan. It's actually not that hard to reconcile."\\
    \normalfont - Jeffrey Flagg \cite{Halstead2019}
    }
\end{center}

Given all of that, we won't be considering neither neo-paganism nor atheist paganism as pagan religions, as
each of them break away, in their own way, from the definition given in the second chapter. Before finishing
this chapter, it's crucial to mention that Satanism, albeit rarely Satanists call themselves pagan, is not a
type of paganism, for it's not only influenced directly by Christianity, but their entire existence relies on
an Abrahamic infrastructure.

\section{Reincarnation and life after death}
Despite the fact that animistic elements can be identified within polytheistic religions to various degrees
(as some pagan religions are more akin to animistic ideas than others), these distinctions often hinge on
individual or cultural conceptions of nature, in contrast to the doctrinal uniformity of Abrahamic traditions.
Yet, perhaps the most profound existential inquiry, both culturally universal and deeply personal, lies in
humanity's enduring fascination with the afterlife.

With thousands of possible interpretations of what happens after death, we will, in this last section, do a
simplification by dividing into two categories of belief: reincarnation and permanent afterlife. In the case
of reincarnation, one of the traditions that are described as following said idea is the Celtic one, as was
described by Julius Caesar on his work "Commentarii de bello Gallico, Liber VI":

\begin{center}
    \itshape
    \parbox{0.7\textwidth}{
    "Druides a bello abesse consuerunt neque tributa una cum reliquis pendunt; militiae vacationem omniumque
    rerum habent immunitatem. Tantis excitati praemiis et sua sponte multi in disciplinam conveniunt et a
    parentibus propinquisque mittuntur. Magnum ibi numerum versuum ediscere dicuntur. Itaque annos nonnulli
    vicenos in disciplina permanent. Neque fas esse existimant ea litteris mandare, cum in reliquis fere rebus,
    publicis privatisque rationibus Graecis litteris utantur. Id mihi duabus de causis instituisse videntur,
    quod neque in vulgum disciplinam efferri velint neque eos, qui discunt, litteris confisos minus memoriae
    studere: quod fere plerisque accidit, ut praesidio litterarum diligentiam in perdiscendo ac memoriam
    remittant. In primis hoc volunt persuadere, non interire animas, sed ab aliis post mortem transire ad alios,
    atque hoc maxime ad virtutem excitari putant metu mortis neglecto. Multa praeterea de sideribus atque eorum
    motu, de mundi ac terrarum magnitudine, de rerum natura, de deorum immortalium vi ac potestate disputant et
    iuventuti tradunt."\\
    \normalfont - Julius Caesar \cite{Caesar-49}
    }
\end{center}

As it can be noticed in the way Julius Caesar writes, the idea of reincarnation was not really a widespread view
among the followers of the Roman religion, rather they believed in a permanent, static afterlife. The dead were
believed to reside in the underworld (Orcus) as shades (manes), and their spirits were honored through rituals
to ensure their peaceful rest and protection of the living. However, a minority, by influence of Greek philosophy,
did indeed adapt some ideas of reincarnation, but it stayed as a niche belief of the elites. Either way, both views
can be accepted under an animistic and idealistic background, apparently with the only difference being that one
view (the reincarnation one) is more restrictive, as in they do kind of limit new consciousness being created by
introducing the idea of reincarnation, while the other view (the permanent afterlife one) is not as much.

In relation to ancestral worship, it does find coherence within idealistic and animistic frameworks, as their
consciousness (the ancestral one) persists as an intrinsic aspect of existence and all entities participate in a
continuum of sentience. Therefore, this continuity suggests that the dead remain present, not as just memories of
the living, but as active and conscious participants.